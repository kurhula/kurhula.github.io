\documentclass[11pt]{beamer}
\usepackage[utf8]{inputenc}
\usepackage[T1]{fontenc}
\usepackage{amsmath}
\usepackage{amsfonts}
\usepackage{amssymb}
\usepackage{graphicx}
\usetheme{default}

\begin{document}
	\author{Musa Baloyi}
	\title{Harnessing the Collective Power of Mobile Devices to Form a Networked Computing Cluster}
	%\subtitle{}
	%\logo{}
	%\institute{}
	%\date{}
	%\subject{}
	%\setbeamercovered{transparent}
	%\setbeamertemplate{navigation symbols}{}
	\begin{frame}[plain]
	\maketitle
\end{frame}

\begin{frame}
\frametitle{Table of contents}
\begin{itemize}
	\item Motivation for research
	\item Volunteer computing projects
	\item Projected contributions
	\item Challenges to mobile computing
	\item Parallel and distributed computing
	\item Schedulers and middleware
	\item Mobile cluster computing
	\item Related work
	\item Case studies
	\item Experimental setup
	\item Architectural setup
	\item Benchmarking libraries	
	\item Software engineering practices
	\item Agile project management
	\item Work to be completed
\end{itemize}
\end{frame}


\begin{frame}
\frametitle{Motivation for research}
\begin{itemize}
	\item Mobile devices: 34m (1993); 5.5b (2011); 6.8b (2013).
	\item Portable, ubiquitous, affordable.
	\item System on Chip (SoC) architecture.
	\item More powerful than desktops of a decade or two ago.
\end{itemize}
\end{frame}

\begin{frame}
\frametitle{Volunteer computing projects}
\begin{itemize}
	\item Great Internet Mersenne Prime Search (GIMPS): for finding the next Mersenne prime
	\item NFS@Home: for performing the lattice sieving step in the Number Field Sieve algorithm
	\item SETI@home: for detecting the availability of extraterrestrial life
	\item distributed.net: for working on different projects
\end{itemize}
\end{frame}

\begin{frame}
\frametitle{Projected contributions}
\begin{itemize}
	\item Developing an architecture for a mobile computing cluster
	\item Implementing BOINC, HTCondor and MOSIX on mobile platforms
	\item Performing benchmark comparisons of BOINC, HTCondor and MOSIX client components on mobile platforms
	\item Providing a number of problem test cases from which more experimentation can follow
\end{itemize}
\end{frame}

\begin{frame}
\frametitle{Challenges to mobile computing}
\begin{itemize}
	\item Their power sustainability is insufficient.
	\item Network connectivity is intermittent.
	\item Modern applications involve big data.
\end{itemize}
\end{frame}

\begin{frame}
\frametitle{Parallel and distributed computing}
\framesubtitle{Open Multi-Processing (OpenMP)}
\begin{figure}[h]
	\centering
	\includegraphics[scale=.3]{../images/parallel.png}
	\caption{Parallel computing}
	\label{parallel_openmp}
\end{figure}
\begin{itemize}
	\item Specification for programming language extensions that allow the software developer to annotate the source code to enable multiple calls to a function or different iterations of a loop to run in parallel.
	\item Concurrent execution inside a single Symmetric Multi-Processing (SMP) or shared-memory computer unit.
\end{itemize}
\end{frame}

\begin{frame}
\frametitle{Parallel and distributed computing}
\framesubtitle{Message Passing Interface (MPI)}
\begin{figure}[h]
	\centering
	\includegraphics[scale=.3]{../images/mpi-arch}
	\caption{Distributed computing}
	\label{distributed_mpi}
\end{figure}
\end{frame}

\begin{frame}
\frametitle{Parallel and distributed computing}
\framesubtitle{Message Passing Interface (MPI)}
\begin{itemize}
	\item Set of interface functions which are callable from user programs written in various languages to do message-passing on Single Program Multiple Data (SPMD) and Multiple Instruction Multiple Data (MIMD) architectures.
	\item The strength of MPI is in running algorithms that require a great deal of interaction, also known as “embarrassingly” (or “massively”) parallel applications.
\end{itemize}
\end{frame}

\begin{frame}
\frametitle{Schedulers and middleware}
\framesubtitle{Berkeley Open Infrastructure for Network Computing (BOINC)}
\begin{figure}[h]
	\centering
	\includegraphics[scale=.3]{../images/how-the-software-works.png}
	\caption{How the BOINC Software Works}
	\label{software}
\end{figure}
\begin{itemize}
	\item installing the software on a server computer,
	\item developing an application,
	\item setting up a project web site,
	\item then distributing the application.
\end{itemize}
\end{frame}

\begin{frame}
\frametitle{Schedulers and middleware}
\framesubtitle{High Throughput Condor (HTCondor)}
\begin{figure}[h]
	\centering
	\includegraphics[scale=.5]{../images/HTCondorArch.png}
	\caption{HTCondor cluster}
	\label{htcondor_cluster}
\end{figure}
\end{frame}

\begin{frame}
\frametitle{Schedulers and middleware}
\framesubtitle{High Throughput Condor (HTCondor)}
\begin{itemize}
	\item Utilizes the computing power of workstations that communicate over a network.
	\item Manage a dedicated cluster of workstations.
	\item Ability to effectively harness non-dedicated, preexisting resources under distributed ownership.
\end{itemize}
\end{frame}

\begin{frame}
\frametitle{Schedulers and middleware}
\framesubtitle{Multicomputer Operating System for UNIX (MOSIX)}
\begin{figure}[h]
	\centering
	\includegraphics[scale=.5]{../images/mosix_cluster.png}
	\caption{A MOSIX multi-cluster}
	\label{mosix_multicluster}
\end{figure}
\end{frame}

\begin{frame}
\frametitle{Schedulers and middleware}
\framesubtitle{Multicomputer Operating System for UNIX (MOSIX)}
\begin{itemize}
	\item Cluster operating system targeted for distributed computing on x86 architecture Linux clusters and multi-clusters.
	\item Dynamic resource discovery and automatic workload distribution
\end{itemize}
\end{frame}

\begin{frame}
\frametitle{Mobile cluster computing}
\begin{figure}[h]
	\centering
	\includegraphics[scale=.2]{../images/Mobile_Cloud_Architecture.jpg}
	\caption{Mobile Cloud Computing}
	\label{mcc1}
\end{figure}
\end{frame}


\begin{frame}
\frametitle{Mobile cluster computing}
\begin{itemize}
	\item The mobile network can be connected via Bluetooth, Wi-Fi or Ethernet (via USB).
	\item The cluster network consists of personal computers connected via Ethernet or Wi-Fi on a Local Area Network (LAN) or Wide Area Network (WAN).
\end{itemize}
\end{frame}


\begin{frame}
\frametitle{Related work}
\begin{itemize}
	\item Basic mobile cloud computing model, which includes a network of mobile devices as well as a cloud platform.
	\item Operating system, abstraction, or implementation, e.g. MAUI, DroidCluster, CloneCloud, etc.
	\item Application of mobile cloud computing, e.g. Vehicular Cloud Computing, Health Monitoring Cloud Computing, Mobile Intelligent Learning System, etc.
\end{itemize}
\end{frame}


\begin{frame}
\frametitle{Related work}
\begin{figure}[h]
	\centering
	\includegraphics[scale=.7]{../images/droidcluster.png}
	\caption{DroidCluster}
	\label{droidcluster}
\end{figure}
\end{frame}

\begin{frame}
\frametitle{Problems addressed}
\begin{itemize}
	\item Matrix multiplication
	\begin{itemize}
		\item Strassen, Cannon, Coppersmith-Winograd, Freivald.
		\item Arithmetic progressions, Monte Carlo, SUMMA, DIMMA, SRUMMA.
		\item Basic Linear Algebra Subprograms (BLAS)
	\end{itemize}
	\item Word frequencies
	\begin{itemize}
		\item String kernels, category-based n-gram language models, class-based n-gram models and statistical algorithms.
		\item Culturomics.
		\item Apache spark word count.
	\end{itemize}
	\item K-means clustering
	\begin{itemize}
		\item Genetic K-means Algorithm (GKA).
		\item Customer segmentation for marketing purposes, online recommender systems.
	\end{itemize}
\end{itemize}
\end{frame}

\begin{frame}
\frametitle{Problems addressed contd.}
\begin{itemize}
	\item Prime factorisation
	\begin{itemize}
		\item Category 1 (special-purpose) algorithms: trial division, wheel factorisation, Pollard’s rho algorithm, Fermat’s factorisation method, Euler’s factorisation method, and special number field sieve.
		\item Category 2 (general-purpose) algorithms: Dixon’s algorithm, continued fraction factorisation, quadratic sieve, rational sieve, and general number field sieve
		\item Crible Algebrique: Distribution, Optimisation - Number Field Sieve  (CADO-NFS).
		\item RSA encryption. RSA Problem. RSA numbers.
		\item The magic words are squeamish ossifrage (AIMS).
	\end{itemize}
\end{itemize}
\end{frame}

\begin{frame}
\frametitle{Experimental setup}
\begin{table}
	\begin{tabular}{| l | l | l | l |}
		\hline
		\textbf{Environment} & \textbf{Middleware} & \textbf{Problem addressed} & \textbf{Paradigm} \\ 
		\hline \hline
		Beowulf cluster & BOINC & Matrix multiplication &  Sequential \\
		Mobile cluster & HTCondor & Word counts &  Parallel \\
		& MOSIX & Prime factors & Distributed \\
		&  & K-means clustering & \\
		\hline 
	\end{tabular}
	\caption{Summary of implementation architectures}
\end{table}
\begin{itemize}
	\item e.g. Mobile cluster + BOINC + matrix multiplication + sequential order.
\end{itemize}
\end{frame}

\begin{frame}
\frametitle{Architectural setup}
\begin{figure}[h]
	\centering
	\includegraphics[scale=.35]{../images/new-hpce-arch}
	\caption{Network and architecture}
	\label{net_arch}
\end{figure}
\end{frame}


\begin{frame}
\frametitle{Servers}
\framesubtitle{https://bitbucket.org/mbaloyi/hpce-servers/src}
\begin{itemize}
	\item Ubuntu 16.04 64-bit (kurhula/vumunhu)
	\item BOINC (kurhula/boinc) or HTCondor (kurhula/htcondor) or MOSIX (kurhula/mosix)
	\item Docker: Dockerfile \& Docker Compose 
	\item boinc-utils (kurhula/boinc-utils)
	\item hpce-common (kurhula/hpce-common)
	\item Bring up and tear down scripts
\end{itemize}
\end{frame}

\begin{frame}
\frametitle{Clients}
\framesubtitle{https://bitbucket.org/mbaloyi/hpce-clients/src}
\begin{itemize}
\item CMake \& Make (CPack)
\item GCC \& Clang
\item MPI \& OpenMP
\item DLib \& GTest
\item Toolchains, e.g. Android, iOS, Windows Mobile.
\item Problem implementations, e.g. matrix multiplication, prime factorisation, n-grams, k-means.
\end{itemize}
\end{frame}

\begin{frame}
\frametitle{Benchmarking libraries}
\begin{itemize}
	\item High Performance Computing Challenge (HPCC) - Top500.org.
	\begin{itemize}
		\item HPL, DGEMM and FFTE collectively measure \textbf{CPU speed}.
		\item STREAM and RandomAccess measure \textbf{memory access and update}, respectively, or memory subsystem, collectively.
		\item PTRANS and b\_eff measure \textbf{bandwidth}.
	\end{itemize}
	\item Beowulf Performance Suite (BPS), NASA Parallel Benchmarks, MPI-IO Test, Gromacs, NetPIPE, Netperf, etc.
\end{itemize}

\end{frame}

\begin{frame}
\frametitle{Software engineering practices}
\begin{figure}[h]
	\centering
	\includegraphics[scale=.2]{../images/bitbucket} \hspace{1mm}
	\includegraphics[scale=.2]{../images/git-logo} \hspace{1mm}
	\includegraphics[scale=.3]{../images/clion2} \hspace{1mm}
	\includegraphics[scale=.1]{../images/eclipse} \vspace{1cm}
	\includegraphics[scale=.05]{../images/iterm2} \hspace{0mm}
	\includegraphics[scale=.15]{../images/oh-my-zsh} \hspace{2mm}
	\includegraphics[scale=.06]{../images/sublime} \hspace{2mm}
	\includegraphics[scale=.6]{../images/texstudio} \hspace{2mm}
	\includegraphics[scale=.03]{../images/vim} 
\end{figure}

\end{frame}

\begin{frame}
\frametitle{Agile project management}
\framesubtitle{Activities and time allocation}
\begin{figure}[h]
	\centering
	\includegraphics[scale=.3]{../images/gantt_chart_agile.png}
	\label{gantt_chart_agile}
\end{figure}
\end{frame}

\begin{frame}
\frametitle{Agile project management}
\framesubtitle{Content checklist}
\begin{figure}[h]
	\centering
	\includegraphics[scale=.2]{../images/trello.png}
	\label{trello}
\end{figure}
\end{frame}

\begin{frame}
\frametitle{Work to be completed}
\begin{itemize}
	\item Implementing the four case studies
	\item Copying the case study onto the mobile devices that form the cluster
	\item Understanding the lifecycles of all three middleware
	\item Running the experiments with different configurations
\end{itemize}
\end{frame}

\begin{frame}
\frametitle{Possible extensions to this work}
\begin{itemize}
	\item Raspberry Pi clusters for educational, demonstration and light applications purposes
	\item In flight, maritime and terrestrial transport mobile clouds and their applications
	\item Drones fitted with system on chips (SoCs) to form clusters for disaster management, monitoring, mining and climate applications
	\item Hardware acceleration of mobile clouds with GPU, Digital Signal Processor (DSP) and cryptographic accelerators
	\item CUDA and OpenCL programming for mobile cluster computing
\end{itemize}
\end{frame}

\end{document}